
\documentclass[10pt]{article}

\usepackage{amsmath, amssymb}
\usepackage{url}
\usepackage{xcolor}
\usepackage{tikz}
\usepackage{mdframed}
\usepackage[papersize={7in,10in}, width=6in, height=8.5in, top=0.75in, ignoreheadfoot]{geometry}
% Number 3:
% \definecolor{bg}{HTML}{DCD0C0}
% \definecolor{txt}{HTML}{333333}
% \definecolor{ltxt}{HTML}{F4F4F4}
% \definecolor{boxbg}{HTML}{333333}
% \definecolor{spine}{HTML}{91a191}
% Number 9:
\definecolor{bg}{HTML}{CEC8B7}
\definecolor{txt}{HTML}{21201E}
\definecolor{ltxt}{HTML}{CEC8B7}
\definecolor{boxbg}{HTML}{21201E}
\definecolor{spine}{HTML}{4C141D}
\definecolor{fcolor}{HTML}{293B44}
\pagecolor{bg}

\usepackage[T1]{fontenc}
\usepackage{newpxtext}
\usepackage[vvarbb,cmintegrals,cmbraces,bigdelims]{newpxmath}
\usepackage[scr=rsfso]{mathalfa}% \mathscr is fancier than \mathcal
\linespread{1.04}         % adds more leading (space between lines)
% quantifiers look strange, so change those back to normal:
	\DeclareSymbolFont{mysymbols}{OMS}{cmsy}{b}{n} %note we make the figures bold to better match newpx.  Replace the ``b'' with an ``m'' to undo this.
	%\SetSymbolFont{mysymbols}  {bold}{OMS}{cmsy}{b}{n}
	%\DeclareSymbolFont{myoperators}   {OT1}{cmr} {m}{n}
	%\SetSymbolFont{myoperators}{bold}{OT1}{cmr} {bx}{n}
	\DeclareMathSymbol{\forall}{\mathord}{mysymbols}{"38}
	\DeclareMathSymbol{\exists}{\mathord}{mysymbols}{"39}
	%\DeclareMathSymbol{\pm}{\mathbin}{mysymbols}{"06}
	%\DeclareMathSymbol{+}{\mathbin}{myoperators}{"2B}
	%\DeclareMathSymbol{-}{\mathbin}{mysymbols}{"00}
	%\DeclareMathSymbol{=}{\mathrel}{myoperators}{"3D}


\begin{document}

\pagestyle{empty}

\color{txt}


% ~
\tikz[remember picture, overlay]{
% \draw[fill, boxbg, thick] (current page.south east) rectangle +(-7in, 3in);
\draw[fill, boxbg, thick] (current page.south east) rectangle ([yshift=3in]current page.south west);

% \draw[line width=10pt, fcolor] (current page.south east) +(0, 3in) -- +(-7in, 3in);
% \draw[fill, boxbg, thick] (current page.north west) rectangle +(8.5in, -.75in);
\draw[fill, spine, thick] (current page.north west) rectangle +(1.25in, -10in);
% \draw[line width=10pt, fcolor] (current page.north west) +(1in,0) -- +(1in, -10in);
}


~
\vskip 1.25in

\begin{flushright}
  \tikz[scale=0.6,yscale=.25]{\draw (0,0) rectangle (1,1);
  \draw (1.4,0) rectangle (3.4, 1) (3.8,0) rectangle (4.8,1) (4.8,0) rectangle (5.8,1); \draw (6.2,0) rectangle (8.2,1) (8.2,0) rectangle (9.2,1) (9.6,0) rectangle (10.6,1) (10.6,0) rectangle (12.6,1) (13,0) rectangle (14,1) (14,0) rectangle (15,1) (15,0) rectangle (16,1);
  }

  \vskip 2em

% \resizebox{\linewidth}{!}{\scshape Exploring Combinatorial Mathematics}
\resizebox{.4\linewidth}{!}{\scshape Exploring}

\vskip 1em
\resizebox{.57\linewidth}{!}{\scshape Combinatorial}

\vskip 1em
\resizebox{.48\linewidth}{!}{\scshape Mathematics}
\vskip 1em

% {\Huge\scshape Exploring}
% 
% \vskip 1em
% {\Huge\scshape Combinatorial}
% 
% \vskip 1em
% {\Huge \scshape Mathematics}
% \vskip 2em
\tikz[scale=0.6,yscale=.25]{\draw (0,0) rectangle (1,1);
\draw (1.4,0) rectangle (3.4, 1) (3.8,0) rectangle (4.8,1) (4.8,0) rectangle (5.8,1); \draw (6.2,0) rectangle (8.2,1) (8.2,0) rectangle (9.2,1) (9.6,0) rectangle (10.6,1) (10.6,0) rectangle (12.6,1) (13,0) rectangle (14,1) (14,0) rectangle (15,1) (15,0) rectangle (16,1);
}

\vskip 2em

% \resizebox{.25\linewidth}{!}{\scshape via Graph Theory}


%\tikz[scale=0.55]{\draw (0,0) rectangle (1,1);
%\draw (1.4,0) rectangle (3.4, 1) (3.8,0) rectangle (4.8,1) (4.8,0) rectangle (5.8,1); \draw (6.2,0) rectangle (8.2,1) (8.2,0) rectangle (9.2,1) (9.6,0) rectangle (10.6,1) (10.6,0) rectangle (12.6,1) (13,0) rectangle (14,1) (14,0) rectangle (15,1) (15,0) rectangle (16,1);
%}

\vskip 3.25in
{
\color{ltxt}
\resizebox{.65\linewidth}{!}{\scshape Richard Grassl \& Oscar Levin}

\vskip .5in

\resizebox{.2\linewidth}{!}{\scshape 1st Edition}
}

\end{flushright}

\clearpage

%\includepdf[pages=-,pagecommand={\thispagestyle{empty}}]{frontmatter/cover2}
%
%

\end{document}






%\addtocontents{toc}{\protect\thispagestyle{plain}}
