\documentclass[11pt]{article}

\usepackage{url}
\usepackage{xcolor}
\usepackage{tikz}
\usepackage{mdframed}
\usepackage[papersize={7in,10in}, hmargin={1in, 2in}, top=.25in, textheight=10in]{geometry}

\definecolor{bg}{HTML}{CEC8B7}
\definecolor{txt}{HTML}{21201E}
\definecolor{ltxt}{HTML}{CEC8B7}
\definecolor{boxbg}{HTML}{21201E}
\definecolor{spine}{HTML}{4C141D}
\definecolor{fcolor}{HTML}{293B44}
\pagecolor{bg}

\usepackage[T1]{fontenc}
\usepackage{newpxtext}
\usepackage[vvarbb,cmintegrals,cmbraces,bigdelims]{newpxmath}
\usepackage[scr=rsfso]{mathalfa}% \mathscr is fancier than \mathcal
\linespread{1.04}         % adds more leading (space between lines)
% quantifiers look strange, so change those back to normal:
	\DeclareSymbolFont{mysymbols}{OMS}{cmsy}{b}{n} %note we make the figures bold to better match newpx.  Replace the ``b'' with an ``m'' to undo this.
	%\SetSymbolFont{mysymbols}  {bold}{OMS}{cmsy}{b}{n}
	%\DeclareSymbolFont{myoperators}   {OT1}{cmr} {m}{n}
	%\SetSymbolFont{myoperators}{bold}{OT1}{cmr} {bx}{n}
	\DeclareMathSymbol{\forall}{\mathord}{mysymbols}{"38}
	\DeclareMathSymbol{\exists}{\mathord}{mysymbols}{"39}
	%\DeclareMathSymbol{\pm}{\mathbin}{mysymbols}{"06}
	%\DeclareMathSymbol{+}{\mathbin}{myoperators}{"2B}
	%\DeclareMathSymbol{-}{\mathbin}{mysymbols}{"00}
	%\DeclareMathSymbol{=}{\mathrel}{myoperators}{"3D}


\begin{document}

\pagestyle{empty}
~

\tikz[remember picture, overlay]{
\draw[fill, boxbg, thick] (current page.north east) rectangle +(-8.5in, -1.25in);
% \draw[fill, boxbg, thick] (current page.north west) rectangle +(8.5in, -.75in);
\draw[fill, spine, thick] (current page.north east) rectangle +(-1in, -11in);
}

% \begin{center}
% % \begin{tikzpicture}[yshift=-.75in, scale=.9, remember picture, overlay, color=bg!75]
% % 	\coordinate (h9) at (0,0);
% % 	\coordinate (h8) at (4,1);
% % 	\coordinate (h7) at (2,1);
% % 	\coordinate (h6) at (0,1);
% % 	\coordinate (h5) at (-2,1);
% % 	\coordinate (h4) at (-4,1);
% % 	\coordinate (h3) at (3,2);
% % 	\coordinate (h2) at (0,2);
% % 	\coordinate (h1) at (-3,2);
% % 	\coordinate (h0) at (0,3);
% %
% % 	\draw[color=bg!75] (h0) -- (h1) -- (h4) -- (h9) -- (h8) -- (h3) -- (h0) -- (h2) -- (h6) -- (h9) -- (h5) -- (h1) -- (h6) -- (h3) -- (h7) -- (h9);
% % 	\foreach \i in {0,...,9}{
% % 	\draw[fill=bg!75, color=bg!75] (h\i) circle (2pt);
% % 	}
% % \end{tikzpicture}
% \end{center}
\hspace{-3em}{\color{ltxt} Mathematics}

\color{txt}
\vskip 1in
\center{\begin{minipage}{4.25in}
\noindent This book was originally written to be used in an MA level course for current secondary math teachers.  Topics have been selected to illustrate larger concepts of interest to secondary teachers, with an emphasis on understanding simple concepts deeply and in more than one way.  Although some topics intersect secondary curriculum, most of the questions here are at a higher level.  Still, the problem solving strategies and big ideas illustrated by our questions have applications to secondary mathematics.  This emphasis is quite different than other mid-level discrete and combinatorics textbooks, since the goal is not to prepare  readers to begin a career in research mathematics.  
\vskip 1em

\noindent Little is assumed about the reader's previous work in the subject, beyond a general understanding of how abstract mathematics proceeds, as well as some basic ability with mathematical proof.  For the reader completely unfamiliar with these and the basic objects of mathematical study (sets and functions), background material is included in an Appendix.
\vskip 1em



\noindent  While the book does not address how to teach mathematics, it tries to model good pedagogical practice.  Almost all of the textbook consists of Activities and Exercises that guide readers to discover mathematics for themselves.  This will require quite a bit more work, both from students and instructors, but the authors strongly believe that the best way to learn mathematics is by doing mathematics.  Most of all, discovering mathematics is fun.
\vskip 1em




\
\end{minipage}
}
\vskip 1em
\begin{center}
The most recent electronic version is available for free at \\ \url{http://discrete.openmathbooks.org/ecm/}
\end{center}

\vskip 7em

\begin{flushleft}
	\color{txt}
	% \resizebox{.25\linewidth}{!}{\textit{Open Math Books}}
	% \resizebox{.25\linewidth}{!}{$\mathbb{O}\mathrm{p}\mathrm{e}\mathrm{n}$ $\mathbb{M}\mathrm{a}\mathrm{t}\mathrm{h}$ $\mathbb{B}\infty\mathrm{k}\mathrm{s}$}
	\resizebox{.2\linewidth}{!}{$\mathbb{O}\mathrm{p}\mathrm{e}\mathrm{n}$ $\mathbb{M}\mathrm{a}\mathrm{t}\mathrm{h}$ $\mathbb{B}\mathrm{oo}\mathrm{k}\mathrm{s}$}
	% \resizebox{.25\linewidth}{!}{$\mathbb{O}{p}{e}{n}$ $\mathbb{M}{a}{t}{h}$ $\mathbb{B}\infty{k}{s}$}
\end{flushleft}
\clearpage

\end{document}
